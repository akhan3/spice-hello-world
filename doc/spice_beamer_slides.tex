\documentclass{beamer}
\usepackage[utf8]{inputenc}
\usepackage{xcolor}
\usepackage{booktabs}
\usepackage{amsmath}
\usepackage{amssymb}

% Title page information
\title{SPICE Simulator Tutorial}
\subtitle{Advanced Analog IC Design - Graduate Level}
\author{ECE 6XX - Advanced Analog IC Design}
\institute{Graduate Course}
\date{\today}

\begin{document}

% Title slide
\begin{frame}
    \titlepage
    \begin{center}
        \textbf{Focus:} Comprehensive SPICE Simulation Types \& Operating Modes \\
        \textbf{Target:} Graduate Students with Advanced Circuit Design Background
    \end{center}
\end{frame}

% Table of contents
\begin{frame}{Outline}
    \tableofcontents
\end{frame}

\section{Introduction to SPICE}

\begin{frame}{Introduction to SPICE}
    \begin{block}{What is SPICE?}
        \textbf{SPICE:} Simulation Program with Integrated Circuit Emphasis
    \end{block}
    
    \subsection{Historical Context \& Versions}
    \begin{itemize}
        \item \textbf{Origin:} University of California, Berkeley
        \item \textbf{Major Versions:} SPICE2G6 and SPICE3F4
        \item \textbf{Commercial Variants:} PSpice, HSPICE, Spectre, etc.
        \item \textbf{Open Source:} Ngspice, WinSpice, etc.
    \end{itemize}
    
    \subsection{Core Algorithm}
    \begin{alertblock}{Modified Nodal Analysis (MNA)}
        Matrix-based circuit solving method that forms the mathematical foundation for all SPICE variants
    \end{alertblock}
\end{frame}

\begin{frame}{Why This Matters for Advanced IC Design}
    \begin{itemize}
        \item Industry standard for analog circuit verification
        \item Essential for understanding simulation limitations and accuracy
        \item Critical for advanced topics like noise, mismatch, and process variation analysis
        \item Foundation for automated design flows and optimization
    \end{itemize}
    
    \begin{exampleblock}{Advanced Applications}
        \begin{itemize}
            \item Statistical yield analysis
            \item Process corner verification
            \item Parasitic extraction validation
            \item Design optimization loops
        \end{itemize}
    \end{exampleblock}
\end{frame}

\section{SPICE Input File Architecture}

\begin{frame}{SPICE Input File Architecture}
    \begin{columns}
        \begin{column}{0.5\textwidth}
            \subsection{Line Types}
            \begin{itemize}
                \item \textbf{Title Line:} First line (any text)
                \item \textbf{Comments:} * in column 1
                \item \textbf{Netlist:} Start with letter
                \item \textbf{Control:} . in column 1
                \item \textbf{Continuation:} + in column 1
            \end{itemize}
        \end{column}
        \begin{column}{0.5\textwidth}
            \subsection{Critical Rules}
            \begin{itemize}
                \item First line is ALWAYS title
                \item Last line must be .end
                \item Case insensitive
                \item Line-oriented parsing
            \end{itemize}
        \end{column}
    \end{columns}
    
    \subsection{Token Separation}
    \begin{itemize}
        \item Separators: spaces, =, ( ), tabs
        \item Numbers: 1.23e-5, 0.0000123
        \item Strings: alphanumeric sequences
        \item Parameters: PARAM=VALUE format
    \end{itemize}
\end{frame}

\begin{frame}{File Structure Best Practices}
    \begin{alertblock}{Pro Tip}
        For advanced simulations, maintain consistent formatting and extensive commenting for complex parameter sweeps and Monte Carlo analyses.
    \end{alertblock}
    
    \subsection{Advanced Considerations}
    \begin{itemize}
        \item Use meaningful variable names for complex circuits
        \item Implement hierarchical commenting structure
        \item Version control integration
        \item Automated netlist generation from scripts
    \end{itemize}
    
    \subsection{Common Pitfalls}
    \begin{itemize}
        \item Forgetting title line requirement
        \item Missing .end statement
        \item Inconsistent node numbering
        \item Case sensitivity issues in model names
    \end{itemize}
\end{frame}

\section{Netlist Construction}

\begin{frame}{Netlist Construction}
    \subsection{Node Labeling Requirements}
    \begin{itemize}
        \item \textbf{Ground Node:} Must be labeled "0"
        \item \textbf{Node Labels:} Positive integers for compatibility
        \item \textbf{Every Node:} Must have a label, even two-terminal connections
    \end{itemize}
    
    \subsection{Circuit Element Types}
    \begin{table}
        \centering
        \begin{tabular}{@{}cll@{}}
            \toprule
            Letter & Element & Syntax Example \\
            \midrule
            V & Voltage Source & V1 1 0 5V \\
            I & Current Source & I1 1 0 1mA \\
            R & Resistor & R1 1 2 1k \\
            C & Capacitor & C1 1 2 1p IC=0V \\
            L & Inductor & L1 1 2 1n IC=0A \\
            D & Diode & D1 1 2 DMODEL \\
            Q & BJT & Q1 c b e QMODEL \\
            M & MOSFET & M1 d g s b MMODEL \\
            T & Transmission Line & T1 1 0 2 0 Z0=50 TD=1ns \\
            \bottomrule
        \end{tabular}
    \end{table}
\end{frame}

\begin{frame}{Advanced Element Considerations}
    \begin{alertblock}{For IC Design}
        Pay special attention to MOSFET models (Level 1-100+), parasitic capacitances, and substrate connections
    \end{alertblock}
    
    \subsection{MOSFET Modeling}
    \begin{itemize}
        \item Substrate connection critical for bulk effects
        \item Model level selection (BSIM3, BSIM4, PSP)
        \item Temperature dependencies
        \item Process variation parameters
    \end{itemize}
    
    \subsection{Parasitic Elements}
    \begin{itemize}
        \item \texttt{M1 d g s sub NMOS W=10u L=0.18u} ; Explicit substrate
        \item \texttt{Cpd d sub 50f} ; Drain-substrate cap
        \item \texttt{Cgs g s 100f} ; Gate-source cap
    \end{itemize}
\end{frame}

\section{Independent Source Specifications}

\begin{frame}{Independent Source Specifications}
    \subsection{Multi-Analysis Source Definition}
    \begin{itemize}
        \item \texttt{V1 1 0 DC\_VALUE AC AC\_MAG AC\_PHASE TRANSIENT\_FUNCTION}
        \item \texttt{I1 1 0 DC\_VALUE AC AC\_MAG AC\_PHASE TRANSIENT\_FUNCTION}
    \end{itemize}
    
    \subsection{DC Analysis Sources}
    \begin{itemize}
        \item \texttt{VDC 1 0 5} ; 5V DC source
        \item \texttt{VDC 1 0 DC 5} ; Explicit DC specification
    \end{itemize}
    
    \subsection{AC Analysis Sources}
    \begin{itemize}
        \item \texttt{VAC 1 0 0 AC 1 0} ; 1V magnitude, 0$^\circ$ phase
        \item \texttt{VAC 1 0 DC 0 AC 1 90} ; 0V DC, 1V AC at 90$^\circ$
    \end{itemize}
\end{frame}

\begin{frame}{Transient Analysis Functions}
    \begin{columns}
        \begin{column}{0.5\textwidth}
            \subsection{Pulse Function}
            \texttt{PULSE V1 V2 TD TR TF PW PER}
            \begin{itemize}
                \item V1: Initial value
                \item V2: Peak value
                \item TD: Delay time
                \item TR: Rise time
                \item TF: Fall time
                \item PW: Pulse width
                \item PER: Period
            \end{itemize}
        \end{column}
        \begin{column}{0.5\textwidth}
            \subsection{Sinusoidal Function}
            \texttt{SIN VO VA FREQ TD THETA}
            \begin{itemize}
                \item VO: Offset voltage
                \item VA: Amplitude
                \item FREQ: Frequency
                \item TD: Delay
                \item THETA: Damping factor
            \end{itemize}
        \end{column}
    \end{columns}
    
    \subsection{Advanced Waveforms}
    \begin{itemize}
        \item PWL (Piecewise Linear) for arbitrary waveforms
        \item EXP (Exponential) for RC responses
        \item SFFM (Single Frequency FM) for modulation analysis
    \end{itemize}
\end{frame}

\section{Device Models}

\begin{frame}{Device Models}
    \subsection{Model Definition Syntax}
    \texttt{.MODEL MODELNAME TYPE (PARAM1=VALUE1 PARAM2=VALUE2 ...)}
    
    \begin{columns}
        \begin{column}{0.5\textwidth}
            \subsection{Semiconductor Models}
            \begin{itemize}
                \item \textbf{D:} Diode models
                \item \textbf{NPN/PNP:} BJT models
                \item \textbf{NMOS/PMOS:} MOSFET models
                \item \textbf{NJF/PJF:} JFET models
            \end{itemize}
        \end{column}
        \begin{column}{0.5\textwidth}
            \subsection{Example Diode Model}
            \begin{itemize}
                \item \texttt{.MODEL DCLAMP D(}
                \item \texttt{+ IS=1e-14}
                \item \texttt{+ RS=0.1}
                \item \texttt{+ CJO=1p}
                \item \texttt{+ VJ=0.7}
                \item \texttt{+ M=0.5)}
            \end{itemize}
        \end{column}
    \end{columns}
\end{frame}

\begin{frame}{Advanced MOSFET Models for IC Design}
    \begin{alertblock}{Industry Models}
        \textbf{BSIM3, BSIM4, PSP, HiSIM} \\
        \textbf{Key Parameters:} VT0, KP, LAMBDA, GAMMA, PHI, TOX, LD, WD \\
        \textbf{Process Variations:} Statistical models, corner models, Monte Carlo parameters
    \end{alertblock}
    
    \begin{itemize}
        \item \texttt{.MODEL NMOS NMOS (LEVEL=49 VERSION=3.3.0}
        \item \texttt{+ TOX=1.4E-8 XJ=1E-7 NCH=2.3549E17 VTH0=0.3691}
        \item \texttt{+ K1=0.5810697 K2=4.774618E-3 K3=0.0431669}
        \item \texttt{+ ... [hundreds of parameters for advanced models])}
    \end{itemize}
    
    \subsection{Model Complexity Considerations}
    \begin{itemize}
        \item Level 1: Simple square-law model
        \item Level 49: BSIM3v3 - Industry standard
        \item Level 54: BSIM4 - Advanced short-channel effects
        \item Statistical parameters for yield analysis
    \end{itemize}
\end{frame}

\section{Analysis Types}

\begin{frame}{DC Analysis}
    \begin{itemize}
        \item \texttt{.DC SRCNAME VSTART VSTOP VINCR [SRC2NAME VSTART2 VSTOP2 VINCR2]}
        \item \texttt{.DC V1 -5 5 0.1} ; Single source sweep
        \item \texttt{.DC V1 0 5 0.1 TEMP -40 125 25} ; Nested sweep with temperature
    \end{itemize}
    
    \subsection{Applications for IC Design}
    \begin{itemize}
        \item Transfer characteristics (VTC curves)
        \item Bias point analysis
        \item Process corner verification
        \item Temperature coefficient analysis
    \end{itemize}
    
    \subsection{Advanced DC Analysis}
    \begin{itemize}
        \item Multi-dimensional parameter sweeps
        \item Operating point extraction
        \item Small-signal parameter calculation
        \item Convergence analysis for complex circuits
    \end{itemize}
\end{frame}

\begin{frame}{AC Analysis}
    \begin{itemize}
        \item \texttt{.AC DEC|OCT|LIN NP FSTART FSTOP}
        \item \texttt{.AC DEC 100 1 1G} ; Logarithmic frequency sweep
        \item \texttt{.AC LIN 1000 1k 100k} ; Linear frequency sweep
    \end{itemize}
    
    \subsection{Critical for Analog IC Design}
    \begin{itemize}
        \item \textbf{Gain-bandwidth analysis}
        \item \textbf{Stability analysis} (phase/gain margins)
        \item \textbf{Input/output impedance}
        \item \textbf{Frequency response optimization}
    \end{itemize}
    
    \begin{exampleblock}{Stability Analysis}
        Use AC analysis with loop breaking to determine:
        \begin{itemize}
            \item Phase margin > 60$^\circ$ for stability
            \item Gain margin > 10dB
            \item Unity gain frequency
        \end{itemize}
    \end{exampleblock}
\end{frame}

\begin{frame}{Transient Analysis}
    \subsection{Syntax and Parameters}
    \texttt{.TRAN TSTEP TSTOP [TSTART [TMAX]] [UIC]}
    
    \texttt{.TRAN 1n 100n 0 0.1n UIC}
    \begin{itemize}
        \item TSTEP: Print step interval
        \item TSTOP: Final time
        \item TSTART: Start time for printing (default 0)
        \item TMAX: Maximum internal timestep
        \item UIC: Use Initial Conditions
    \end{itemize}
    
    \begin{columns}
        \begin{column}{0.5\textwidth}
            \subsection{Timestep Selection}
            \begin{itemize}
                \item \textbf{Rule of thumb:} TSTEP $\leq$ Period/100
                \item \textbf{High-speed circuits:} Consider rise/fall times
                \item \textbf{Oscillators:} Multiple periods for settling
            \end{itemize}
        \end{column}
        \begin{column}{0.5\textwidth}
            \subsection{Applications}
            \begin{itemize}
                \item Settling time analysis
                \item Slew rate measurements
                \item Power-up sequences
                \item Clock jitter analysis
            \end{itemize}
        \end{column}
    \end{columns}
\end{frame}

\begin{frame}{Initial Conditions and Advanced Transient}
    \subsection{Initial Conditions}
    \begin{itemize}
        \item \texttt{C1 1 2 1p IC=2.5V} ; Capacitor initial voltage
        \item \texttt{L1 1 2 1n IC=0A} ; Inductor initial current
        \item \texttt{.IC V(3)=1.2V V(4)=0V} ; Node initial conditions
    \end{itemize}
    
    \begin{alertblock}{Advanced Tip}
        Use .IC for complex bias networks, UIC for charge-sharing analysis
    \end{alertblock}
    
    \subsection{Transient Optimization}
    \begin{itemize}
        \item Adaptive timestep algorithms
        \item Breakpoint insertion for discontinuities
        \item Interpolation methods for output
        \item Memory management for long simulations
    \end{itemize}
\end{frame}

\section{Advanced Analysis Types}

\begin{frame}{Noise Analysis}
    \begin{itemize}
        \item \texttt{.NOISE V(OUTPUT) VIN DEC 100 1 1G}
        \item \texttt{.NOISE V(5) VIN OCT 20 1k 1MEG 1}
    \end{itemize}
    
    \begin{itemize}
        \item \textbf{Output node:} Where noise is measured
        \item \textbf{Input source:} Reference for input-referred noise
        \item \textbf{Applications:} LNA design, ADC noise analysis, oscillator phase noise
    \end{itemize}
    
    \subsection{Noise Sources in IC Design}
    \begin{itemize}
        \item Thermal noise (4kTR)
        \item Shot noise (2qI)
        \item Flicker noise (1/f)
        \item Induced gate noise in MOSFETs
    \end{itemize}
    
    \begin{exampleblock}{Noise Optimization}
        Critical for low-noise amplifiers, precision references, and high-resolution converters
    \end{exampleblock}
\end{frame}

\begin{frame}{Monte Carlo and Sensitivity Analysis}
    \subsection{Monte Carlo Analysis}
    \begin{itemize}
        \item \texttt{.MC 100 TRAN 1n 100n YMAX V(out)} ; 100 runs, find max V(out)
        \item \texttt{.PARAM R1\_VAR=GAUSS(1k,50,3)} ; Gaussian distribution
    \end{itemize}
    
    \subsection{Sensitivity Analysis}
    \begin{itemize}
        \item \texttt{.SENS V(5)} ; DC sensitivity
        \item \texttt{.SENS V(out,5k)} ; AC sensitivity at 5kHz
    \end{itemize}
    
    \begin{alertblock}{Process Variation Analysis}
        \textbf{Statistical Models:} Mismatch parameters, process corners \\
        \textbf{Key Applications:} Yield optimization, design centering, worst-case analysis
    \end{alertblock}
    
    \subsection{Advanced Statistical Methods}
    \begin{itemize}
        \item Latin Hypercube Sampling
        \item Importance sampling for rare events
        \item Response surface modeling
        \item Design of experiments (DOE)
    \end{itemize}
\end{frame}

\section{Output Control and Data Extraction}

\begin{frame}{Output Control and Data Extraction}
    \subsection{Print Statements}
    \begin{itemize}
        \item \texttt{.PRINT DC V(1) V(2) I(VDD)} ; DC print
        \item \texttt{.PRINT AC VM(5) VP(5) VDB(5)} ; AC magnitude, phase, dB
        \item \texttt{.PRINT TRAN V(out) I(R1) P(M1)} ; Transient variables
    \end{itemize}
    
    \begin{columns}
        \begin{column}{0.5\textwidth}
            \subsection{Mathematical Functions}
            \begin{itemize}
                \item \texttt{VDB(5)} ; 20*log10(|V(5)|)
                \item \texttt{VP(5)} ; Phase of V(5)
                \item \texttt{VM(5)} ; Magnitude of V(5)
                \item \texttt{VR(5)} ; Real part of V(5)
                \item \texttt{VI(5)} ; Imaginary part of V(5)
            \end{itemize}
        \end{column}
        \begin{column}{0.5\textwidth}
            \subsection{Device Parameters}
            \begin{itemize}
                \item \texttt{@M1[gm]} ; Transconductance
                \item \texttt{@M1[gds]} ; Output conductance
                \item \texttt{@M1[id]} ; Drain current
                \item \texttt{@M1[vgs]} ; Gate-source voltage
                \item \texttt{@M1[vth]} ; Threshold voltage
            \end{itemize}
        \end{column}
    \end{columns}
\end{frame}

\begin{frame}{Modern Graphical Output}
    \begin{alertblock}{Graphical Post-Processing}
        \textbf{PSpice:} .PROBE statement generates data for Probe viewer \\
        \textbf{Other SPICE variants:} Automatic generation of plot data files \\
        \textbf{Post-processing:} Python/MATLAB scripts for custom analysis
    \end{alertblock}
    
    \subsection{Advanced Data Processing}
    \begin{itemize}
        \item Automated measurement functions
        \item Statistical data analysis
        \item Multi-dimensional plotting
        \item Export to standard formats (CSV, HDF5)
    \end{itemize}
    
    \subsection{Measurement Examples}
    \begin{itemize}
        \item \texttt{.MEAS TRAN trise TRIG V(out) VAL=0.1 RISE=1}
        \item \texttt{+ TARG V(out) VAL=0.9 RISE=1}
        \item \texttt{.MEAS AC bandwidth WHEN VDB(out)=-3}
    \end{itemize}
\end{frame}

\section{Operating Modes}

\begin{frame}{SPICE Operating Modes}
    \subsection{Batch Mode Operation}
    \begin{itemize}
        \item \texttt{spice3 -b input.cir > output.txt} \# Background batch
        \item \texttt{pspice input.cir} \# PSpice batch
        \item \texttt{hspice input.sp} \# HSPICE batch
    \end{itemize}
    
    \subsection{Interactive Mode}
    \begin{itemize}
        \item \texttt{spice3} \# Start interactive session
        \item \texttt{spice 1 -> source input.cir} \# Load circuit
        \item \texttt{spice 2 -> run} \# Execute analyses
        \item \texttt{spice 3 -> setplot tran1} \# Select analysis
        \item \texttt{spice 4 -> plot v(out)} \# Generate plot
        \item \texttt{spice 5 -> print v(out)} \# Print values
        \item \texttt{spice 6 -> quit} \# Exit
    \end{itemize}
\end{frame}

\begin{frame}{Interactive Commands}
    \begin{columns}
        \begin{column}{0.5\textwidth}
            \subsection{Analysis Control}
            \begin{itemize}
                \item \textbf{run:} Execute analyses
                \item \textbf{stop:} Halt simulation
                \item \textbf{resume:} Continue simulation
                \item \textbf{reset:} Clear all data
            \end{itemize}
        \end{column}
        \begin{column}{0.5\textwidth}
            \subsection{Data Manipulation}
            \begin{itemize}
                \item \textbf{setplot:} Choose analysis dataset
                \item \textbf{display:} Show available variables
                \item \textbf{let:} Define new variables
                \item \textbf{alter:} Modify circuit parameters
            \end{itemize}
        \end{column}
    \end{columns}
    
    \subsection{Advanced Interactive Features}
    \begin{itemize}
        \item Real-time parameter modification
        \item Interactive debugging
        \item Custom function definition
        \item Script execution within interactive mode
    \end{itemize}
\end{frame}

\section{GUI vs Command Line}

\begin{frame}{GUI vs Command Line Interfaces}
    \begin{columns}
        \begin{column}{0.5\textwidth}
            \subsection{GUI Advantages}
            \begin{itemize}
                \item Visual schematic entry
                \item Interactive simulation control
                \item Real-time plotting
                \item Parameter sweeping tools
                \item Built-in calculators
            \end{itemize}
        \end{column}
        \begin{column}{0.5\textwidth}
            \subsection{Popular GUI Tools}
            \begin{itemize}
                \item \textbf{Cadence Virtuoso}
                \item \textbf{Mentor Graphics ADS}
                \item \textbf{Synopsys HSPICE}
                \item \textbf{LTspice} (free)
                \item \textbf{Ngspice + KiCad}
            \end{itemize}
        \end{column}
    \end{columns}
    
    \begin{alertblock}{Command Line Benefits}
        \textbf{Automation:} Scripting, batch processing, parameter optimization \\
        \textbf{Precision:} Exact parameter control, repeatable simulations \\
        \textbf{Integration:} Design flows, version control, continuous integration \\
        \textbf{Performance:} Faster execution, lower memory overhead
    \end{alertblock}
\end{frame}

\begin{frame}{Best Practices for IC Design}
    \subsection{Strategic Tool Usage}
    \begin{itemize}
        \item \textbf{Initial Design:} Use GUI for rapid prototyping and visualization
        \item \textbf{Verification:} Command line for automated corner analysis
        \item \textbf{Optimization:} Scripted parameter sweeps and yield analysis
        \item \textbf{Documentation:} Netlist-based approach for reproducibility
    \end{itemize}
    
    \subsection{Integration Strategies}
    \begin{itemize}
        \item Version control systems (Git, SVN)
        \item Continuous integration pipelines
        \item Automated regression testing
        \item Design database management
    \end{itemize}
    
    \begin{exampleblock}{Professional Workflow}
        Combine GUI for design entry with command-line automation for verification and optimization
    \end{exampleblock}
\end{frame}

\section{Practical Example}

\begin{frame}[fragile]{Complete SPICE Example: Diode Clamp Circuit}
    \begin{alertblock}{Circuit Description}
        \textbf{Application:} Voltage clamping circuit with RC time constant analysis \\
        \textbf{Components:} Pulse source, capacitor, diode, load resistor
    \end{alertblock}
    
    \begin{verbatim}
SPICE sample circuit - diode clamp
*comment lines begin with asterisks
*independent voltage source with DC value, AC value, and
*transient square wave value:
V1 1 0 1 AC 1 pulse -10 20 0 1.e-8 1.e-8 1e-3 2e-3
*the square wave defined above has -10V to +20V extent, with
*a period of 2 milliseconds
*capacitor for clamping
C1 1 2 1e-6
*diode for clamp - model name is dclamp
D1 2 0 dclamp
*load resistor - large enough that RC >> 2 ms
R1 2 0 1e5
*model for diode
.model dclamp D(IS=1e-14)
    \end{verbatim}
\end{frame}

\begin{frame}[fragile]{Diode Clamp Example - Analysis Commands}
    \begin{verbatim}
*DC transfer function generated for this circuit
.DC V1 -20 20 .1
*AC frequency sweep - assumes circuit is biased with V1 = 1V
.AC OCT 20 10 1e2 1e4
*frequency is swept logarithmically from 100Hz to 10000Hz
*transient analysis will show clamping
.TRAN 1e-4 8e-3 0 1e-5
*start at time zero, go for 8 ms, make internal steps 10 microsec
*save print data at .1 ms intervals
*that's all folks
.end
    \end{verbatim}
    
    \subsection{Analysis Insights}
    \begin{itemize}
        \item DC analysis reveals clamping threshold
        \item AC analysis shows frequency response limitations
        \item Transient analysis demonstrates dynamic clamping behavior
        \item Time constants validate RC >> switching period
    \end{itemize}
\end{frame}

\section{Advanced SPICE Features}

\begin{frame}{Hierarchical Design}
    \begin{itemize}
        \item \texttt{.SUBCKT OPAMP 1 2 3 4 5} ; Non-inv, Inv, Out, VDD, VSS
        \item \texttt{* Internal circuitry here}
        \item \texttt{.ENDS OPAMP}
        \item \texttt{X1 in1 in2 out vdd vss OPAMP} ; Instantiate subcircuit
    \end{itemize}
    
    \subsection{Parameter Passing}
    \begin{itemize}
        \item \texttt{.SUBCKT DIFFPAIR 1 2 3 4 PARAMS: W=10u L=0.18u}
        \item \texttt{M1 3 1 5 4 NMOS W=\{W\} L=\{L\}}
        \item \texttt{M2 3 2 6 4 NMOS W=\{W\} L=\{L\}}
        \item \texttt{.ENDS}
        \item \texttt{X1 in+ in- out vss DIFFPAIR W=20u L=0.25u}
    \end{itemize}
    
    \subsection{Advanced Hierarchy}
    \begin{itemize}
        \item Multi-level subcircuit nesting
        \item Global parameter propagation
        \item Conditional circuit topology
        \item Library management
    \end{itemize}
\end{frame}

\begin{frame}{Process Corners and Statistical Analysis}
    \subsection{Process Corners}
    \begin{itemize}
        \item \texttt{.LIB '/path/to/models' TT} ; Typical-Typical
        \item \texttt{.LIB '/path/to/models' FF} ; Fast-Fast
        \item \texttt{.LIB '/path/to/models' SS} ; Slow-Slow
        \item \texttt{.LIB '/path/to/models' SF} ; Slow-Fast
        \item \texttt{.LIB '/path/to/models' FS} ; Fast-Slow
    \end{itemize}
    
    \begin{alertblock}{Statistical Analysis}
        \textbf{Mismatch Analysis:} Device-to-device variations within the same die \\
        \textbf{Process Variations:} Die-to-die and wafer-to-wafer variations \\
        \textbf{Environmental Stress:} Temperature, voltage, aging effects
    \end{alertblock}
    
    \subsection{Corner Analysis Strategy}
    \begin{itemize}
        \item Systematic corner verification
        \item Statistical yield prediction
        \item Design centering optimization
        \item Robustness validation
    \end{itemize}
\end{frame}

\section{Simulation Accuracy and Convergence}

\begin{frame}{Convergence Options}
    \begin{itemize}
        \item \texttt{.OPTIONS GMIN=1e-12} ; Minimum conductance
        \item \texttt{.OPTIONS ABSTOL=1e-12} ; Absolute current tolerance
        \item \texttt{.OPTIONS RELTOL=1e-3} ; Relative tolerance
        \item \texttt{.OPTIONS VNTOL=1e-6} ; Voltage tolerance
        \item \texttt{.OPTIONS ITL1=100} ; DC iteration limit
        \item \texttt{.OPTIONS ITL2=50} ; DC transfer curve iteration limit
        \item \texttt{.OPTIONS ITL4=10} ; Transient iteration limit
    \end{itemize}
    
    \begin{columns}
        \begin{column}{0.5\textwidth}
            \subsection{Common Issues}
            \begin{itemize}
                \item Poor initial guesses
                \item Floating nodes
                \item Large signal discontinuities
                \item Unrealistic device models
                \item Numerical precision limits
            \end{itemize}
        \end{column}
        \begin{column}{0.5\textwidth}
            \subsection{Solutions}
            \begin{itemize}
                \item Add .NODESET statements
                \item Use .IC for difficult bias points
                \item Ramp sources gradually
                \item Check model parameters
                \item Adjust solver tolerances
            \end{itemize}
        \end{column}
    \end{columns}
\end{frame}

\begin{frame}{Advanced Solver Options}
    \begin{itemize}
        \item \texttt{.OPTIONS METHOD=GEAR} ; Integration method
        \item \texttt{.OPTIONS MAXORD=2} ; Maximum integration order
        \item \texttt{.OPTIONS PIVREL=1e-3} ; Relative pivot threshold
        \item \texttt{.OPTIONS PIVTOL=1e-13} ; Absolute pivot threshold
    \end{itemize}
    
    \subsection{Numerical Methods}
    \begin{itemize}
        \item \textbf{GEAR:} Implicit multi-step method for stiff systems
        \item \textbf{TRAPEZOIDAL:} Second-order accurate, A-stable
        \item \textbf{EULER:} First-order, simple but less accurate
    \end{itemize}
    
    \begin{exampleblock}{Convergence Strategy}
        Start with relaxed tolerances, then tighten for accuracy. Use source ramping and initial conditions for difficult circuits.
    \end{exampleblock}
\end{frame}

\section{Industry Best Practices}

\begin{frame}{Simulation Strategy}
    \begin{columns}
        \begin{column}{0.5\textwidth}
            \subsection{Design Phase Approach}
            \begin{enumerate}
                \item \textbf{Hand Calculations} - Initial sizing
                \item \textbf{Simplified Models} - Topology verification
                \item \textbf{Detailed Models} - Performance optimization
                \item \textbf{Corner Analysis} - Robustness verification
                \item \textbf{Statistical Analysis} - Yield prediction
            \end{enumerate}
        \end{column}
        \begin{column}{0.5\textwidth}
            \subsection{Verification Hierarchy}
            \begin{enumerate}
                \item \textbf{Unit Cells} - Individual stages
                \item \textbf{Functional Blocks} - Combined stages
                \item \textbf{System Level} - Complete circuits
                \item \textbf{Layout Parasitic} - Post-layout verification
                \item \textbf{Silicon Correlation} - Model validation
            \end{enumerate}
        \end{column}
    \end{columns}
    
    \begin{alertblock}{Documentation and Version Control}
        \textbf{Simulation Decks:} Maintain versioned, commented netlist libraries \\
        \textbf{Model Management:} Centralized, validated device model libraries \\
        \textbf{Results Archival:} Automated result storage with simulation metadata
    \end{alertblock}
\end{frame}

\begin{frame}{Performance Optimization}
    \subsection{Computational Efficiency}
    \begin{itemize}
        \item \textbf{Parallel Simulation:} Multi-core utilization for corner analysis
        \item \textbf{Incremental Updates:} Only re-simulate changed portions
        \item \textbf{Checkpoint/Restart:} Resume long simulations efficiently
        \item \textbf{Smart Convergence:} Adaptive solver settings
    \end{itemize}
    
    \subsection{Automation Strategies}
    \begin{itemize}
        \item Scripted parameter sweeps
        \item Automated measurement extraction
        \item Statistical post-processing
        \item Design optimization loops
    \end{itemize}
    
    \begin{exampleblock}{Modern Workflows}
        Integration with machine learning for design space exploration and automated optimization
    \end{exampleblock}
\end{frame}

\section{Summary}

\begin{frame}{Summary and Key Takeaways}
    \begin{alertblock}{Critical Knowledge for Advanced IC Design}
        \textbf{SPICE Fundamentals:} Understanding MNA, convergence, and accuracy limitations \\
        \textbf{Analysis Types:} DC, AC, Transient, Noise, Monte Carlo applications \\
        \textbf{Operating Modes:} Strategic use of GUI vs. command line approaches
    \end{alertblock}
    
    \begin{columns}
        \begin{column}{0.5\textwidth}
            \subsection{Technical Competencies}
            \begin{itemize}
                \item Advanced device modeling
                \item Statistical analysis interpretation
                \item Convergence troubleshooting
                \item Performance optimization
            \end{itemize}
        \end{column}
        \begin{column}{0.5\textwidth}
            \subsection{Professional Skills}
            \begin{itemize}
                \item Simulation methodology
                \item Documentation standards
                \item Version control practices
                \item Cross-platform compatibility
            \end{itemize}
        \end{column}
    \end{columns}
\end{frame}

\begin{frame}{Next Steps}
    \begin{enumerate}
        \item \textbf{Hands-on Practice:} Implement complex analog circuits using multiple SPICE variants
        \item \textbf{Advanced Topics:} Noise optimization, mismatch analysis, yield modeling
        \item \textbf{Integration:} SPICE in automated design flows and optimization loops
        \item \textbf{Research Applications:} Custom models, algorithm development, emerging devices
    \end{enumerate}
    
    \begin{alertblock}{Remember}
        SPICE is a tool to validate and optimize your designs, not replace fundamental analog design intuition and understanding.
    \end{alertblock}
    
    \begin{center}
        \Large{Questions and Discussion}
    \end{center}
\end{frame}

\end{document}